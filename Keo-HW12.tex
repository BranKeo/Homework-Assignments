%%%%%%%%%%%%%%%%%%%%%%%%%%%%%%%%%%%%%%%%%%%%%%%%%%%%%%%%%%%%
%%%%%%%%%%%%%%%%%%%%%%%%%%%%%%%%%%%%%%%%%%%%%%%%%%%%%%%%%%%%
%%%%%%%%%%%%%%%%%%%%%%%%%%%%%%%%%%%%%%%%%%%%%%%%%%%%%%%%%%%%
%%%%%%%%%%%%%%%%%%%%%%%%%%%%%%%%%%%%%%%%%%%%%%%%%%%%%%%%%%%%
%%%%%%%%%%%%%%%%%%%%%%%%%%%%%%%%%%%%%%%%%%%%%%%%%%%%%%%%%%%%
\documentclass[12pt]{article}
\usepackage{epsfig}
\usepackage{times}
\usepackage{amsmath}
\renewcommand{\topfraction}{1.0}
\renewcommand{\bottomfraction}{1.0}
\renewcommand{\textfraction}{0.0}
\setlength {\textwidth}{6.6in}
\hoffset=-1.0in
\oddsidemargin=1.00in
\marginparsep=0.0in
\marginparwidth=0.0in                                                                               
\setlength {\textheight}{9.0in}
\voffset=-1.00in
\topmargin=1.0in
\headheight=0.0in
\headsep=0.00in
\footskip=0.50in                                         
\setcounter{page}{1}
\begin{document}
\def\pos{\medskip\quad}
\def\subpos{\smallskip \qquad}
\newfont{\nice}{cmr12 scaled 1250}
\newfont{\name}{cmr12 scaled 1080}
\newfont{\swell}{cmbx12 scaled 800}
%%%%%%%%%%%%%%%%%%%%%%%%%%%%%%%%%%%%%%%%%%%%%%%%%%%%%%%%%%%%
%     DO NOT CHANGE ANYTHING ABOVE THIS LINE
%%%%%%%%%%%%%%%%%%%%%%%%%%%%%%%%%%%%%%%%%%%%%%%%%%%%%%%%%%%%
%     DO NOT CHANGE ANYTHING ABOVE THIS LINE
%%%%%%%%%%%%%%%%%%%%%%%%%%%%%%%%%%%%%%%%%%%%%%%%%%%%%%%%%%%%
%     DO NOT CHANGE ANYTHING ABOVE THIS LINE
%%%%%%%%%%%%%%%%%%%%%%%%%%%%%%%%%%%%%%%%%%%%%%%%%%%%%%%%%%%%

\begin{center}
\normalsize
\bf {PHYS  20323/60323: Fall 2020 - LaTeX Example}
%%%%%%%%%%%%%%%%%%%%%%%%%%%%%%%%%%%%%%%%%%%%%%%%%%%%%%%%%%%%
\end{center}
\vskip.1in
%%%%%%%%%%%%%%%%%%%%%%%%%%%%%%%%%%%%%%%%%%%%%%%%%%%%%%%%%%%%

% Basically an Authors List.
\noindent{1. Consider a particle confined in a two-dimensional infinite square well}\\

%%%%%%%%%%%%%%%%%%%%%%%%%%%%%%%%%%%%%%%%%%%%%%%%%%%%%%%%%%%%
% Section Heading
%%%%%%%%%%%%%%%%%%%%%%%%%%%%%%%%%%%%%%%%%%%%%%%%%%%%%%%%%%%%
\vskip0.1in
\begin{center}
\noindent
$V(x,y) = \begin{cases}
0, & 0 \leq x \leq a, 0 < y < a\\
\infty, & \textrm{otherwise}
\end{cases}
$
\end{center}\\

\noindent The eigenfunctions have the form:

\begin{center}
\large $\Psi(x,y) = \frac{2}{a}~sin~(\frac{n\pi x}{a})~sin~(\frac{m\pi y }{a})$
\end{center}

\noindent with the corresponding energies given by:

\begin{center}
\large $E_n_m = (n^2 + m^2)~\frac{\pi^2\hbar^2}{2ma^2}$
\end{center}

\noindent (a) (5 points) What are the levels of degeneracy of the five lowest energy values?
\vskip.03in
\noindent (b) (5 points) Consider a perturbation given by:
\begin{center}
\large $\hat{H}' = a^2~V_0~\delta(x-\frac{a}{2})~\delta(y-\frac{a}{2})$
\end{center}

Calculate the first order correction to the ground state energy.\\


\vskip0.1in
\noindent {2. \bf The following questions refer to stars in the Table below.}\\
Note: There may be multiple answers.
\vskip.1in


%%%%%%%%%%%%%%%%%%%%%%%%%%%%%%%%%%%%%%%%%%%%%%%%%%%%%%%%%%%%
% Tables are created easily
%%%%%%%%%%%%%%%%%%%%%%%%%%%%%%%%%%%%%%%%%%%%%%%%%%%%%%%%%%%%
\noindent
\begin{tabular}{|l|c|c|c|c|c|}\hline
Name & Mass & Luminosity & Lifetime & Temperature & Radius \\\hline
Zeta & 60. M_s_u_n & 10^6 L_s_u_n & 8.0 x 10^5~years &  &     \\\hline
Epsilon & 6.0 M_s_u_n & 10^3 L_s_u_n &  & 20,000 K &     \\\hline
Delta & 2.0 M_s_u_n & & 5 x 10^8~years & & 2 R_s_u_n \\\hline
Beta & 1.3 M_s_u_n & 3.5 L_s_u_n & & &  \\\hline
Alpha & 1.0 M_s_u_n & & & & 1 R_s_u_n \\\hline
Gamma & 0.7 M_s_u_n & & 4.5 x 10^10~years & 5000 K & \\\hline


\end{tabular}\vskip 0.2in


\noindent (a) (4 points) Which of these stars will produce a planetary nebula at the end of their life.
\vskip .4in
\noindent (b) (4 points) Elements heavier than \textit{Carbon} will be produced in which stars.

\end{document}
